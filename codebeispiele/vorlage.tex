\documentclass{scrartcl}
\usepackage[ngerman]{babel}
\usepackage[T1]{fontenc}
\usepackage[utf8]{inputenc}
\usepackage{geometry}
\geometry{a4paper, top=2.5cm, left=2.5cm, right=2.5cm, bottom=2.5cm}
\usepackage{scrlayer-scrpage}
\usepackage[breaklinks,colorlinks,linkcolor=black,citecolor=black,filecolor=black,urlcolor=black]{hyperref}

\title{Beispieldokument für \LaTeX}
\author{Autor}
\date{\today}
\begin{document}
\maketitle
\tableofcontents
\newpage

\section{Einleitung}
Ein paar einleitende Sätze, die einen Überblick über das Dokument geben.

\section{Hauptteil}
\subsection{Der erste wichtige Punkt}
Es gibt wirklich erstaunliche Fakten.

\subsection{Ein weiterer wichtiger Punkt}
Dieser Punkt existiert nur, damit es nicht einen einzelnen Unterpunkt gibt.

\end{document}
