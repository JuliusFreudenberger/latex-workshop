\documentclass[presentation,aspectratio=169]{beamer}
\usepackage[utf8]{inputenc}
\usepackage[T1]{fontenc}
\usepackage{graphicx}
\usepackage[ngerman]{babel}
\usepackage{longtable,capt-of,fvextra,csquotes}
\usepackage{wrapfig,rotating}
\usepackage[normalem]{ulem}
\usepackage{amsmath,amssymb}
\usepackage{hyperref}
\usepackage{minted}
\usepackage[duration=20]{pdfpc}
\usetheme{metropolis}
\author{Julius Freudenberger}
\date{Hackathon Sommersemester 2022}
\title{WYSIWYAF with \LaTeX}
\mode<presentation>{\usetheme{metropolis}}
\mode<beamer|handout>{\metroset{sectionpage=progressbar}}
\mode<beamer|handout>{\metroset{subsectionpage=progressbar}}
\mode<beamer|handout>{\metroset{block=fill}}
\institute[Hochschule Esslingen]{Hochschule Esslingen}
\hypersetup{
  pdfauthor={Julius Freudenberger},
  pdftitle={WYSIWYAF with LaTeX},
  pdfkeywords={},
  pdfsubject={},
pdflang={German}}
\usepackage{biblatex}

\begin{document}

\maketitle

\begin{frame}[fragile]{Bevor wir beginnen: Was brauche ich?}
  \begin{itemize}
    \item \LaTeX-Distribution: \TeX{}Live
      \begin{itemize}
        \item Windows: \href{https://tug.org/texlive/windows.html}{https://tug.org/texlive/windows.html} %TODO MikTeX?
        \item Mac: \href{https://tug.org/mactex/}{https://tug.org/mactex/}
        \item Linux: Installation über den Paketmanager
          \begin{itemize}
            \item deb: \verb|texlive-base| (deb), \verb|texlive texlive-latex| (rpm)
            \item Arch Linux: \verb|texlive-core|
            \item NixOS: \verb|nixpkgs.texlive.combined.scheme-basic|
          \end{itemize}
        \item Docker: \verb|texlive/texlive|
      \end{itemize}
    \item Texteditor
      \begin{itemize}
        \item VSCode mit \LaTeX-Workshop, vim mit vimtex
        \item \TeX{}Maker
      \end{itemize}
    \item Alternativ: Online-Editoren
      \begin{itemize}
        \item Overleaf (\href{https://www.overleaf.com}{https://www.overleaf.com}, Registrierung erforderlich)
        \item \TeX{}Viewer (\href{https://texviewer.herokuapp.com}{https://texviewer.herokuapp.com}, direkt nutzbar)
      \end{itemize}
  \end{itemize}
\end{frame}

\begin{frame}{Was ist \LaTeX?}
  \pdfpcnote{ - Entwicklung seit Anfang der 1980er
  - What you see is what you get --> What you see is what you asked for}
  \begin{itemize}
    \item Textsatzsystem
    \item setzt vorgegebenen Text und weitere Anweisungen automatisch
    \item versucht automatisch bestmögliches Layout
    \item kein WYSIWYG, sondern WYSIWYAF
    \item Quellcode wird \glqq{}kompiliert\grqq{}
    \item Dokument wird als PDF, PS, DVI oder sogar HTML ausgegeben
  \end{itemize}
\end{frame}

\begin{frame}{Warum \LaTeX?}
  \pdfpcnote{ Vergleich zu Word
    - schöner Blocksatz, bestmögliche Zeilenumbrüche und Worttrennungen
    - Inhaltsverzeichnis und Verzeichnisse für Abbildungen, Stichworte, Abkürzungen und Literatur
    - Abbildungen und weitere Referenzelemente werden automatisch an der bestmöglichen Stelle eingefügt
  - Deutlich mehr Automatisierungen als bei Word, "Layoutarbeit" am Ende einer Arbeit deutlich geringer}
  \begin{itemize}
    \item Automatischer Textsatz
    \item Automatische Referenzen
    \item Automatisches Layout
  \end{itemize}
\end{frame}

\begin{frame}{Wie sieht \LaTeX-Code aus?}
  \pdfpcnote{
    - Befehle werden mit \ angegeben
    - Parameter mit {}
    - Einrückung hilft bei Lesbarkeit, ist nicht notwendig
    - Beginn und Ende
    - normaler Text
  }
  \inputminted{latex}{codebeispiele/beispiel.tex}
\end{frame}

\begin{frame}{Grundlegender Aufbau}
  \pdfpcnote{
    - Dokumentenklasse (book, article, beamer)
    - Präambel
    - Metadaten (Titel, Autor, Datum)
    - Zusätzliche Pakete
    - Einstellungen
    - Eigentliches Dokument
  }
  \inputminted{latex}{codebeispiele/aufbau.tex}
\end{frame}

\begin{frame}{Vorlage herunterladen}
  \begin{itemize}
    \item \href{https://www2.hs-esslingen.de/~jufrit00/latex/}{https://www2.hs-esslingen.de/\textasciitilde{}jufrit00/latex/}
    \item \href{https://gitlab.hs-esslingen.de/jufrit00/latex-workshop}{https://gitlab.hs-esslingen.de/jufrit00/latex-workshop}
    \item \href{https://github.com/JuliusFreudenberger/latex-workshop}{https://github.com/JuliusFreudenberger/latex-workshop}
  \end{itemize}
\end{frame}

\begin{frame}[fragile]{Projekt kompilieren}
  \begin{itemize}
    \item Im Projektverzeichnis \verb|pdflatex file.tex|
    \item automatisierter mit \verb|latexmk -pdf file.tex|
    \item In \TeX{}Maker \glqq{}Schnelles Übersetzen\grqq{}
    \item mittels Docker und Docker Compose: %TODO
    \item Outputfile \verb|file.pdf| als PDF-Datei im gleichen Verzeichnis
  \end{itemize}
\end{frame}

\begin{frame}[fragile]{Erste eigene Änderungen}
  \pdfpcnote{
    - Stärke von LaTeX: Änderungen wirken sich konsistent auf das gesamte Dokument aus.
    - Keine manuelle Anpassung an mehreren Stellen nötig.
    - Niedrigste Stufe: subsubsection
    - Bei Text am Besten ein Satz in eine Zeile.
  }
  \begin{itemize}
    \item Eigener Text, Absätze
    \item Textformatierung: \textbf{fett} und \textit{kursiv} mit \verb|\textbf{fett}|, \verb|\textit{kursiv}|
    \item Eigene Abschnitte mit \verb|\section{} und \subsection{}|
    \item Metadaten ändern
      \begin{itemize}
        \item Titel des Dokuments ändern mit \verb|\title{}|
        \item Eigener Name als Autor mit \verb|\author{}|
      \end{itemize}
  \end{itemize}
\end{frame}

\begin{frame}[fragile]{Umgebungen}
  \begin{itemize}
    \item Umgebungen beginnen mit \verb|\begin{}| und enden mit \verb|\end{}|
    \item Umgebungen können verschachtelt werden, aber nicht geschnitten
    \item Aufzählungen \verb|enumerate| (nummeriert) und \verb|itemize| (unnummeriert)
  \end{itemize}
  \begin{center}
    \centering
    \begin{minted}{latex}
\begin{itemize}
  \item Erster Punkt
  \item Zweiter Punkt
\end{itemize}
    \end{minted}
  \end{center}
\end{frame}
\end{document}
