\documentclass[presentation,aspectratio=169]{beamer}
\usepackage[utf8]{inputenc}
\usepackage[T1]{fontenc}
\usepackage{graphicx}
\graphicspath{{./graphics/}}
\usepackage[ngerman]{babel}
\usepackage{longtable,capt-of,fvextra,csquotes}
\MakeOuterQuote{"}
\usepackage{wrapfig,rotating}
\usepackage[normalem]{ulem}
\usepackage{amsmath,amssymb}
\usepackage{hyperref}
\usepackage{listings,minted}
\usepackage[duration=20]{pdfpc}
\usetheme{metropolis}
\author{Julius Freudenberger}
\date{Cyberweek Sommersemester 2025}
\title{WYSIWYAF with \LaTeX}
\subtitle{Wissenschaftliche Arbeiten oder Präsentationen mit \LaTeX -- Fortgeschrittene}
\mode<presentation>{\usetheme{metropolis}}
\mode<beamer|handout>{\metroset{sectionpage=progressbar}}
\mode<beamer|handout>{\metroset{subsectionpage=progressbar}}
\mode<beamer|handout>{\metroset{progressbar=frametitle}}
\mode<beamer|handout>{\metroset{block=fill}}
\institute[Hochschule Esslingen]{Hochschule Esslingen}
\hypersetup{
  pdfauthor={Julius Freudenberger},
  pdftitle={WYSIWYAF with LaTeX -- Aufbaukurs},
  pdfkeywords={},
  pdfsubject={},
  pdflang={German},
  colorlinks,
  linkcolor=black,
  citecolor=black,
  filecolor=black,
  urlcolor=blue
}
\usepackage{biblatex}

\begin{document}

\maketitle

\begin{frame}[fragile]{Bevor wir beginnen: Was brauche ich?}
  \begin{itemize}
    \item \LaTeX-Distribution: \TeX{}Live oder Mik\TeX{}
      \begin{itemize}
        \item Windows: \href{https://miktex.org/}{https://miktex.org/}
        \item Mac: \href{https://tug.org/mactex/}{https://tug.org/mactex/}
        \item Linux: Installation über den Paketmanager
          \begin{itemize}
            \item deb: \verb|texlive-base|, rpm: \verb|texlive texlive-latex|
            \item Arch Linux: \verb|texlive-core|
            \item NixOS: \verb|nixpkgs.texliveBasic|
          \end{itemize}
        \item Docker: \verb|texlive/texlive|
      \end{itemize}
    \item Texteditor
      \begin{itemize}
        \item \href{https://code.visualstudio.com/}{VSCode} mit \href{https://marketplace.visualstudio.com/items?itemName=James-Yu.latex-workshop}{\LaTeX-Workshop}, \href{https://www.vim.org/}{vim} mit \href{https://github.com/lervag/vimtex}{vimtex}
        \item \href{https://www.xm1math.net/texmaker/index.html}{\TeX{}Maker}
      \end{itemize}
    \item Alternativ: Online-Editoren
      \begin{itemize}
        \item Overleaf (\href{https://www.overleaf.com}{https://www.overleaf.com}, Registrierung erforderlich)
        \item \TeX{}Viewer (\href{https://texviewer.herokuapp.com}{https://texviewer.herokuapp.com}, direkt nutzbar)
      \end{itemize}
  \end{itemize}
\end{frame}

\begin{frame}{Über mich}
  \textbf{B.\,Eng. Julius Freudenberger}
  \begin{itemize}
    \item 2000 geboren
    \item 2016 erste Kontakte mit \LaTeX
    \item 2019 -- 2023 Bachelorstudium an der Hochschule Esslingen
    \item 2023 -- 2026(?) Masterstudium an der Hochschule Karlsruhe
    \item nebenbei Werkstudent
  \end{itemize}
\end{frame}

\begin{frame}{Wer seid ihr?}
  \begin{itemize}
    \item Name
    \item Fakultät? Was macht ihr da?
    \item Schon von \LaTeX{} gehört oder etwas damit gemacht?
    \item Was stört dich am meisten an traditionellen Textverarbeitungsprogrammen?
    \item Nächste (wissenschaftliche) Arbeit, die du vielleicht mit \LaTeX{} schreiben möchtest?
  \end{itemize}
\end{frame}

\begin{frame}{Materialien für den Kurs}
  \begin{itemize}
    \item Dieser Foliensatz
    \item Kleine Vorlagen zum Mitmachen
    \item Dokumente, die wir im Laufe des Kurses erstellen
      \bigskip
    \item wird alles im geteilten Ordner hochgeladen
  \end{itemize}
\end{frame}

\begin{frame}{Inhalt}
  \tableofcontents
\end{frame}

\begin{frame}[fragile]{Weitere Verzeichnisse}
  \begin{itemize}
    \item automatisch aktualisierendes Inhaltsverzeichnis sowie Verzeichnisse für Abbildungen, Codelistings (Abkürzungen, Stichwörter, Bibliographie, \dots)
  \end{itemize}
  \inputminted{latex}{codebeispiele/list-of-everything.tex}
\end{frame}

\begin{frame}[fragile]{Verzeichnisse im Inhaltsverzeichnis}
  \inputminted{latex}{codebeispiele/lists-in-toc.tex}
\end{frame}

\section{Präsentationen}

\begin{frame}[fragile]{Präsentationen}
  \begin{itemize}
    \item Dokumentenklasse \verb|beamer|
    \item Präsentation besteht aus einer Abfolge von Folien
      \inputminted{latex}{codebeispiele/beamer-frame.tex}
  \end{itemize}
\end{frame}

\begin{frame}[fragile]{Präsentationen -- Besonderheiten}
  \begin{itemize}
    \item \verb|\section| oder \verb|\subsection| erstellen einzelne Folien mit den Überschriften
    \item \verb|\maketitle| erstellt Titelfolie
    \item zu langer Inhalt verschiebt sich nicht automatisch auf die nächste Folie
    \item Für Listings oder \verb|verbatim| muss der Frame als \verb|fragile| gekennzeichnet werden
      \inputminted{latex}{codebeispiele/frame-fragile.tex}
  \end{itemize}
\end{frame}

\begin{frame}[fragile]{"Animationen" -- Overlays}
  \begin{itemize}
    \item PDF kann keine bewegten Animationen
    \item aber Aufzählungspunkte oder anderer Inhalt kann nacheinander eingeblendet werden
    \item jedes Overlay wird auf einer neuen PDF-Seite angezeigt
      \inputminted{latex}{codebeispiele/beamer-overlays.tex}
  \end{itemize}
\end{frame}

\begin{frame}[fragile]{Themes}
  \begin{itemize}
    \item Klassische Themes in der \href{https://hartwork.org/beamer-theme-matrix/}{Beamer Theme Matrix}
    \item Modernes Theme (auch in dieser Präsentation): \href{https://ctan.org/pkg/beamertheme-metropolis}{Metropolis}
      \begin{itemize}
        \item Moderne Font (wenn mit Xe\LaTeX{} oder Lua\LaTeX{} kompiliert wird)
        \item kann optional einen Fortschrittsbalken anzeigen
        \item Einstellungen mit \verb|\metroset|
      \end{itemize}
  \end{itemize}
\end{frame}

\begin{frame}[fragile]{Bewerbungsunterlagen}
  \begin{columns}
    \column{.5\textwidth}
      \centering
      \begin{itemize}
        \item Lebenslauf mit \verb|moderncv|
      \end{itemize}
      \includegraphics[height=.7\textheight]{Lebenslauf}
    \column{.5\textwidth}
      \centering
      \begin{itemize}
        \item Anschreiben mit \verb|dinbrief|
      \end{itemize}
      \includegraphics[height=.7\textheight]{Anschreiben}
  \end{columns}
\end{frame}

\begin{frame}[fragile]{Wohin bei Problemen?}
  \begin{itemize}
    \item \verb|texdoc <paketname>|
    \item Suchmaschine des Vertrauens
    \item \href{https://de.wikibooks.org/wiki/LaTeX-Kompendium}{\LaTeX-Kompendium}
    \item \href{https://www.overleaf.com/learn}{Overleaf documentation}
    \item \href{https://golatex.de/wiki/index.php/Hauptseite}{Go\LaTeX-Wiki}
    \item \href{https://tex.stackexchange.com}{\TeX-Stackexchange}
  \end{itemize}
\end{frame}

\maketitle

\end{document}
